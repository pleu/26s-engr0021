%----------------------------------------------------------------------------------------
%    PACKAGES AND THEMES
%----------------------------------------------------------------------------------------

\documentclass[aspectratio=169,xcolor=dvipsnames]{beamer}
\usetheme{SimplePlus}

\usepackage{hyperref}
\usepackage{graphicx} % Allows including images
\usepackage{booktabs} % Allows the use of \toprule, \midrule and \bottomrule in tables
\usepackage{etoolbox}
\usepackage{amsmath}

% Variable for whether answers are shown. True = answers shown.
\newbool{answers}
% \booltrue{answers}
\boolfalse{answers}

%----------------------------------------------------------------------------------------
%    TITLE PAGE
%----------------------------------------------------------------------------------------

\AtBeginSection[]{
    \begin{frame}
        \frametitle{}
        \centering
        \huge{\textbf{\insertsection} \textcolor{structure}{}}
    \end{frame}
    }

\title{ENGR 0021 Recitation 1}
\subtitle{January 16th, 2026}

\author{Rie Huntington}

\institute
{
    Department of Industrial Engineering \\
    University of Pittsburgh
}

\date{}
%----------------------------------------------------------------------------------------
%    PRESENTATION SLIDES
%----------------------------------------------------------------------------------------

\begin{document}

\begin{frame}
    \titlepage
\end{frame}

\begin{frame}[t]{Chapter 2, Slide 34 (Submit this solution to Canvas)}
    The number of combinations of $n$ distinct objects taken $r$ at a time is:
    \[\binom{n}{r} = \frac{n!}{r!(n - r)!}\]
    Suppose there are 8 men and 8 women. How many ways can we choose a committee that has 2 women and 2 men? \\

    \vspace{0.5 cm}

    \ifbool{answers}{
        \begin{alertenv}
            \textbf{Answer:} \\
            Ways to pick 2 men: $\binom{8}{2} = 28$ \\
            Ways to pick 2 women: $\binom{8}{2} = 28$ \\
            Since selecting men and women are separate operations, we can use the multiplication rule and multiply the independent choices together to find the total number of ways 2 men and 2 women can be selected:
            \[28 \cdot 28 = 784 \text{ ways}\]
        \end{alertenv}
    }{}
\end{frame}

\begin{frame}[t]{Question 1}
    Calculate the mean, median, mode, variance, standard deviation, and range of the following list sampled from a population: 0, 1, 2, 2, 4, 5, 7 \\

    \vspace{0.5 cm}

    \ifbool{answers}{
        \begin{alertenv}
            \textbf{Answer:}
            \[\text{mean: } \bar{x} = \frac{0+1+2+2+4+5+7}{7} = 3\]
            \[\text{median: }\tilde{x} = 2\]
            \[\text{mode: } 2\]
            \[\text{sample variance: }s^2 = \sum_{i = 1}^{7} \frac{(x_i - \bar{x})^2}{n-1} = \sum_{i = 1}^{7} \frac{(x_i - 3)^2}{6} = 6\]
            \[\text{sample standard deviation: }s = \sqrt{s^2} = \sqrt{6}\]
            \[\text{range: } 7 - 0 = 0\]
        \end{alertenv}
    }{}
\end{frame}

\begin{frame}[t]{Question 2}
    Suppose we have a list with 5 entries. Each entry can either be 1, 2, or 3. What must the list be if the average is 1? If the average is 3? Can the average be 4? \\

    \vspace{0.5 cm}

    \ifbool{answers}{
        \begin{alertenv}
            \textbf{Answer:}
            \begin{itemize}
                \item If the average is 1, all entries must be 1. As 1 is the smallest possible entry, any entry greater than 1 in the list would make the average greater than 1.
                \item If the average is 3, all entries must be 3. As 3 is the largest possible entry, any entry less than 3 in the list would make the average less than 3.
                \item The average cannot be 4. The average of a list is the sum of the entries divided by the number of entries. As the largest possible entry is 3, the maximum sum of 5 entries is 15. 15 divided by 5 is 3, thus the largest the average can be is 3.
            \end{itemize}
        \end{alertenv}
    }{}
\end{frame}

\begin{frame}[t]{Question 3}
    Twenty people in a room have an average height of 65 inches. A twenty-first person enters the room. How tall must they be to raise the average height by one inch? \\

    \vspace{0.5 cm}

    \ifbool{answers}{
        \begin{alertenv}
            \textbf{Answer:}
            Let $h_i$ be the height of person $i$. We know that:
            \[\frac{\sum_{i = 1}^{20} h_i}{20} = 65\]
            This means that:
            \[\sum_{i = 1}^{20} h_i = 65 \cdot 20 = 1300\]
            Given this result, we can solve the following equation to find the height of the 21st person:
            \[\frac{(\sum_{i = 1}^{20} h_i) + h_{21}}{21} = 66 \rightarrow h_{21} = 66 \cdot 21 - \sum_{i = 1}^{20} h_i = 66 \cdot 21 -1300 = 86\]
        \end{alertenv}
    }{}
\end{frame}

\begin{frame}[t]{Question 4}
    Examine the following boxplot: \\
    \begin{figure}
        \centering
        \includegraphics[width=0.6\linewidth]{boxplot_example.png}
    \end{figure}
    Determine the min, max, median, lower fourth, and upper fourth. Are there any outliers?

    \vspace{0.5 cm}
    
    \ifbool{answers}{
        \begin{alertenv}
            \textbf{Answer:}
            The min is 10, the max is 35, the median is 21, the lower fourth is 15, and the upper fourth is 28. There are no outliers (if there were outliers, there would be dots on either side of the min and max stems).
        \end{alertenv}
    }{}
\end{frame}

\begin{frame}[t]{Question 5}
    Given the sample space $S = \{1, 2, 3, 4, 5, 6, 7, 8\}$ and sets $A = \{1, 2, 3, 4, 5\}$, $B = \{1, 3, 5, 7\}$, $C = \{2, 5, 8\}$, find: $A', \ A \cup C, \ A \cap B \cap C \cap S, \ A' \cap B, \ (A \cup B') \cap C'$. \\

    \vspace{0.5 cm}

    \ifbool{answers}{
        \begin{alertenv}
            \textbf{Answer:}
            \begin{enumerate}
                \item $A' = \{6, 7, 8\}$
                \item $A \cup C = \{1, 2, 3, 4, 5, 8\}$
                \item $A \cap B \cap C \cap S = \{5\}$
                \item $A' \cap B = \{7\}$
                \item $(A \cup B') \cap C' = \{1, 3, 4, 6\}$
            \end{enumerate}
        \end{alertenv}
    }{}
\end{frame}

\begin{frame}[t]{Question 6}
    Suppose you are in charge of selecting officers for a club with 10 members. The club needs to elect a President, Vice President, Business Manager, and Secretary. No student can hold more than one office. How many different ways can these officers be chosen?

    \vspace{0.5 cm}

    \ifbool{answers}{
        \begin{alertenv}
            \textbf{Answer:}
            Since the roles are distinct and no student can hold more than one office, the order the officers are selected in matters. For this reason, we can use the permutation formula where $n = 10$ and $r = 4$:
            \[{}_{10}P_{4} = \frac{10!}{(10-4)!} = \frac{10!}{6!} = 10 \cdot 9 \cdot 8 \cdot 7 = 5040\]
        \end{alertenv}
    }{}
\end{frame}

\end{document}
